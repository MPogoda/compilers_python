% vim:spelllang=ru,en
\documentclass[a4paper,10pt,notitlepage,pdftex]{scrreprt}

\usepackage{cmap} % чтобы работал поиск по PDF
\usepackage[utf8]{inputenc}
\usepackage[english,russian]{babel}
\usepackage[T2A]{fontenc}
\usepackage{concrete}

\usepackage[pdftex]{graphicx}
\usepackage{subfig}

\pdfcompresslevel=9 % сжимать PDF
\usepackage{pdflscape} % для возможности альбомного размещения некоторых страниц
\usepackage[pdftex]{hyperref}
% настройка ссылок в оглавлении для pdf формата
\hypersetup{unicode=true,
    pdftitle={Подмножество языка},
    pdfauthor={Погода Михаил},
    pdfcreator={pdflatex},
    pdfsubject={},
    pdfborder    = {0 0 0},
    bookmarksopen,
    bookmarksnumbered,
    bookmarksopenlevel = 2,
    pdfkeywords={},
    colorlinks=true, % установка цвета ссылок в оглавлении
    citecolor=black,
    filecolor=black,
    linkcolor=black,
    urlcolor=blue
}

\usepackage{amsmath}
\usepackage{amssymb}

\usepackage{listings}
\lstloadlanguages{Python,C++}
\lstset{language=Python,frame=tb,basicstyle=\small,commentstyle=\itshape,extendedchars=false}

\usepackage{multicol}

\begin{document}
\begin{titlepage}
    \begin{center}
        \MakeUppercase{Министерство образования и науки Украины}\\
        \MakeUppercase{Национальный технический университет Украины}\\
        \MakeUppercase{<<Киевский политехнический институт>>}\\
        \vspace*{2em}

        Факультет прикладной математики\\
        Кафедра прикладной математики

        \vfill

        \MakeUppercase{Построение синтаксического анализатора}\\
        \vspace*{2em}

        Отчёт по лабораторной работе \textnumero~2\\
        по дисциплине:\\
        <<Системы автоматизации программирования>>
    \end{center}

    \vfill
    \hfill\begin{minipage}{0.3\textwidth}
        Выполнил студент\\
        группы КМ-31м\\
        Погода~М.\,В.
    \end{minipage}

    \vfill
    \begin{center}
        Киев\\
        2014
    \end{center}
\end{titlepage}

\tableofcontents

\chapter{Постановка задачи}
\label{chap:first}
    В рамках этой лабораторной работы необходимо спроектировать синтаксический анализатор для подмножества языка
    программирования \texttt{Python}, которая была описана в предыдущей работе.

    Необходимо использовать алгоритм левой рекурсии (LL(1)).
    Синтаксический LL"=анализатор --- нисходящий синтаксический анализатор для подмножества контекстно"=свободных
    грамматик.
    Он анализирует входной поток слева направо, и строит левый вывод грамматики.
\chapter{Техническое задание}
\label{chap:second}
    \section{Формальное определение исходной грамматики}
    \label{sec:formal}
        Исходная грамматика формально описывается четвёркой:
        \begin{equation}
            \label{eq:grammar}
            G = \left( N, \Sigma, P, S \right)
        \end{equation}
        , де
        \begin{itemize}
            \item $N$ --- алфавит нетерминальных символов (см. стр.~\pageref{para:nonterm});
            \item $\Sigma$ --- алфавит терминальных символов, содержащий лексемы, полученные путём лексического
                анализа (см. стр.~\pageref{para:term});
            \item $P$ --- множество правил синтаксического разбора (см. стр.~\pageref{para:rules});
            \item $S$ --- начальный символ грамматики (\verb'{start}').
        \end{itemize}

        \paragraph{Нетерминальные символы грамматики}
        \label{para:nonterm}
        \begin{multicols}{3}
            \begin{enumerate}
                \item \verb'{applicable}'
                \item \verb'{assignment}'
                \item \verb'{class-decl}'
                \item \verb'{class-list}'
                \item \verb'{classes}'
                \item \verb'{comparator-eq}'
                \item \verb'{comparator-int}'
                \item \verb'{elseline}'
                \item \verb'{expr-int}'
                \item \verb'{fcall}'
                \item \verb'{ifline}'
                \item \verb'{input}'
                \item \verb'{logic-bool}'
                \item \verb'{logic-int}'
                \item \verb'{logic-str}'
                \item \verb'{logic}'
                \item \verb'{mcall}'
                \item \verb'{method-decl}'
                \item \verb'{method-list}'
                \item \verb'{methods}'
                \item \verb'{mparams}'
                \item \verb'{mparam-list}'
                \item \verb'{operand-int}'
                \item \verb'{operand-bool}'
                \item \verb'{operand-str}'
                \item \verb'{operator-int}'
                \item \verb'{param-list}'
                \item \verb'{params}'
                \item \verb'{rightside}'
                \item \verb'{sline-list}'
                \item \verb'{slines}'
                \item \verb'{sline}'
                \item \verb'{start}'
                \item \verb'{whileline}'
            \end{enumerate}
        \end{multicols}

        \paragraph{Терминальные символы грамматики}
        \label{para:term}
        \begin{multicols}{3}
            \begin{enumerate}
                \item \verb'('
                \item \verb')'
                \item \verb'+'
                \item \verb'-'
                \item \verb'*'
                \item \verb'/'
                \item \verb':'
                \item \verb'.'
                \item \verb'='
                \item \verb'<'
                \item \verb'>'
                \item \verb'!'
                \item \verb'def'
                \item \verb'class'
                \item \verb'if'
                \item \verb'else'
                \item \verb'while'
                \item \verb'break'
                \item \verb'retur'
                \item \verb'print'
                \item \verb'input'
                \item \verb'{d-const}' --- вещественное число.
                \item \verb'{b-const}' --- \verb'True' / \verb'False'.
                \item \verb'{s-const}' --- любая строка.
                \item \verb'{identifier}' --- любой идентификатор.
                \item \verb'{newline}' --- конец строки.
                \item \verb'{indent}' --- увеличение отступа.
                \item \verb'{dedent}' --- уменьшение отступа.
                \item $\varepsilon$
            \end{enumerate}
        \end{multicols}

        \paragraph{Правила исходной грамматики}
        \label{para:rules}
            Правило представляются в виде $A \rightarrow B$.
            \begin{enumerate}
                \item \verb'{applicable} ' $\rightarrow$ \verb' {fcall}'
                \item \verb'{applicable} ' $\rightarrow$ \verb' {identifier}'

                \stepcounter{enumi}

                \item \verb'{assignment} ' $\rightarrow$ \verb' {identifier} = {rightside}'
                \item \verb'{class-decl}'$\rightarrow$\verb'class{identifier}:{newline}{indent}{methods}{dedent}'
                \item \verb'{class-list} ' $\rightarrow$ \verb' {classes}'
                \item \verb'{class-list} ' $\rightarrow$ \verb' {eps}'
                \item \verb'{classes} ' $\rightarrow$ \verb' {class-decl} {class-list}'
                \item \verb'{comparator-eq} ' $\rightarrow$ \verb' !='
                \item \verb'{comparator-eq} ' $\rightarrow$ \verb' =='
                \item \verb'{comparator-int} ' $\rightarrow$ \verb' !='
                \item \verb'{comparator-int} ' $\rightarrow$ \verb' <'
                \item \verb'{comparator-int} ' $\rightarrow$ \verb' =='
                \item \verb'{comparator-int} ' $\rightarrow$ \verb' >'
                \item \verb'{elseline} ' $\rightarrow$ \verb' else : {newline} {indent} {slines} {dedent}'
                \item \verb'{elseline} ' $\rightarrow$ \verb' {eps}'
                \item \verb'{expr-int} ' $\rightarrow$ \verb' {operand-int}'
                \item \verb'{expr-int} ' $\rightarrow$ \verb' {operand-int} {operator-int} {operand-int}'
                \item \verb'{fcall} ' $\rightarrow$ \verb' {identifier} ( {params} )'
                \item \verb'{ifline}'$\rightarrow$

                    \verb'if{logic}:{newline}{indent}{slines}{dedent}{elseline}'
                \item \verb'{input} ' $\rightarrow$ \verb' input ( )'
                \item \verb'{logic-bool} ' $\rightarrow$ \verb' {operand-bool}'
                \item \verb'{logic-bool} ' $\rightarrow$ \verb' {operand-bool} {comparator-eq} {operand-bool}'
                \item \verb'{logic-int} ' $\rightarrow$ \verb' {operand-int} {comparator-int} {operand-int}'
                \item \verb'{logic-str} ' $\rightarrow$ \verb' {operand-str} {comparator-eq} {operand-str}'
                \item \verb'{logic} ' $\rightarrow$ \verb' {logic-bool}'
                \item \verb'{logic} ' $\rightarrow$ \verb' {logic-int}'
                \item \verb'{logic} ' $\rightarrow$ \verb' {logic-str}'
                \item \verb'{mcall} ' $\rightarrow$ \verb' {applicable} . {fcall}'
                \item \verb'{method-decl} ' $\rightarrow$

                    \verb'def{identifier}({mparams}):{newline}{indent}{slines}{dedent}'
                \item \verb'{method-list} ' $\rightarrow$ \verb' {eps}'
                \item \verb'{method-list} ' $\rightarrow$ \verb' {methods}'
                \item \verb'{methods} ' $\rightarrow$ \verb' {method-decl} {method-list}'
                \item \verb'{mparams} ' $\rightarrow$ \verb' {identifier} {mparam-list}'
                \item \verb'{mparam-list} ' $\rightarrow$ \verb' , {mparams}'
                \item \verb'{mparam-list} ' $\rightarrow$ \verb' {eps}'
                \item \verb'{operand-int} ' $\rightarrow$ \verb' {identifier}'
                \item \verb'{operand-int} ' $\rightarrow$ \verb' {int-const}'
                \item \verb'{operand-bool} ' $\rightarrow$ \verb' {identifier}'
                \item \verb'{operand-bool} ' $\rightarrow$ \verb' {bool-const}'
                \item \verb'{operand-str} ' $\rightarrow$ \verb' {identifier}'
                \item \verb'{operand-str} ' $\rightarrow$ \verb' {str-const}'
                \item \verb'{operator-int} ' $\rightarrow$ \verb' *'
                \item \verb'{operator-int} ' $\rightarrow$ \verb' +'
                \item \verb'{operator-int} ' $\rightarrow$ \verb' -'
                \item \verb'{operator-int} ' $\rightarrow$ \verb' /'
                \item \verb'{param-list} ' $\rightarrow$ \verb' , {rightside} {param-list}'
                \item \verb'{param-list} ' $\rightarrow$ \verb' {eps}'
                \item \verb'{params} ' $\rightarrow$ \verb' {rightside} { param-list}'
                \item \verb'{params} ' $\rightarrow$ \verb' {eps}'
                \item \verb'{rightside} ' $\rightarrow$ \verb' {expr-int}'
                \item \verb'{rightside} ' $\rightarrow$ \verb' {input}'
                \item \verb'{rightside} ' $\rightarrow$ \verb' {logic}'
                \item \verb'{rightside} ' $\rightarrow$ \verb' {mcall}'
                \item \verb'{rightside} ' $\rightarrow$ \verb' {fcall}'
                \item \verb'{rightside} ' $\rightarrow$ \verb' {str-const}'
                \item \verb'{sline-list} ' $\rightarrow$ \verb' {eps}'
                \item \verb'{sline-list} ' $\rightarrow$ \verb' {slines}'
                \item \verb'{slines} ' $\rightarrow$ \verb' {sline} {sline-list}'
                \item \verb'{sline} ' $\rightarrow$ \verb' break {newline}'
                \item \verb'{sline} ' $\rightarrow$ \verb' return {rightside} {newline}'
                \item \verb'{sline} ' $\rightarrow$ \verb' print ( {rightside} ) {newline}'
                \item \verb'{sline} ' $\rightarrow$ \verb' {assignment} {newline}'
                \item \verb'{sline} ' $\rightarrow$ \verb' {ifline}'
                \item \verb'{sline} ' $\rightarrow$ \verb' {mcall} {newline}'
                \item \verb'{sline} ' $\rightarrow$ \verb' {whileline}'
                \item \verb'{start} ' $\rightarrow$ \verb' {indent} {classes} {mcall} {newline} {dedent}'
                \item \verb'{whileline} ' $\rightarrow$

                    \verb' while {logic}: {newline} {indent} {slines} {dedent}'
            \end{enumerate}
\chapter{Описание логической структуры программы}
\label{chap:third}
    Для синтаксического анализа используется LL(1) алгоритм.
    В качестве исходных данных используются лексемы, которые были выделены с помощью лексического анализатора.

    Эти лексемы подвергаются небольшой предварительной обработке: в нужные места добавляется лексема \verb'{dedent}',
    которая характеризует уменьшение отступа по сравнению с предыдущей строкой.
    Этот шаг был добавлен так как рассматриваемый язык использует отступы в качестве границ семантических блоков.

    Для корректной работы LL(1) парсера необходимо заранее построить таблицу переходов, показывающую, какие правила
    возможно применить и заменить текущий нетерминальный символ на вершине стека.
    Поскольку не всегда есть возможность однозначно определить номер правила, была использована стратегия возврата:
    если есть несколько правил, которые можно применить согласно таблице, то синтаксический анализатор пытается
    разобрать цепочку символов, используя каждое из доступных правил поочерёдно.
    Если во время разбора возникает ошибка, то это означает, что правило было выбрано некорректно и следует
    воспользоваться другим.
    Если же синтаксические ошибки обнаруживаются при применении всех доступных в данной ситуации правил, то это
    означает, что с этой позиции нет правильного разбора в рамках грамматики.

    В качестве результата работы программное обеспечение выдаёт либо последовательность правил, которые были применены
    при успешном разборе, либо сообщение об ошибке.

\chapter{Тестовые примеры}
\label{chap:fourth}
    \section{Пример без синтаксических ошибок}
    \label{sec:ex1}
        \lstinputlisting{../test1.py}
        Вывод программы:
        \[
            66 \rightarrow 8 \rightarrow 5 \rightarrow 33 \rightarrow 30 \rightarrow 34 \rightarrow 35 \rightarrow 34 \rightarrow 36 \rightarrow 58 \rightarrow 63 \rightarrow 20 \rightarrow 27 \rightarrow 24 \rightarrow \dots
        \]

    \section{Пример с синтаксической ошибкой}
    \label{sec:ex2}
        \begin{lstlisting}
            class class class class:
        \end{lstlisting}
\chapter{Выводы}
\label{chap:concl}
    Лексемы, полученные на фазе лексического анализа, используются для построения синтаксической свёртки.
    В результате этой операции получаем либо синтаксический разбор исходного текста, либо уведомление про ошибку.

    В данной работе был реализован синтаксический анализатор на базе LL(1) грамматики, показаны основные свойства
    этого алгоритма, приведены корректный и некорректный примеры.

    Стоит отметить, что в данной работе в классический алгоритм LL(1) была добавлена возможность отката, когда
    существует неопределённость в таблице переходов.
    Для нейтрализации этого неклассического поведения можно использовать LL(k) ($k > 1$) грамматику , так чтобы не
    было неоднозначностей при переходах.
    Однако этот метод требует значительно большего объёма работ для построения таблицы переходов (которая в этом
    случае будет многомерной).
\chapter*{Приложение. Исходный текст}
    \lstset{language=C++,basicstyle=\scriptsize}
    \lstinputlisting{../syntax.h}
\end{document}
