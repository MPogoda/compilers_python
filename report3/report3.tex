% vim:spelllang=ru,en
\documentclass[a4paper,12pt,notitlepage,pdftex]{scrreprt}

\usepackage{cmap} % чтобы работал поиск по PDF
\usepackage[utf8]{inputenc}
\usepackage[english,russian]{babel}
\usepackage[T2A]{fontenc}
\usepackage{concrete}

\usepackage[pdftex]{graphicx}
\usepackage{subfig}

\pdfcompresslevel=9 % сжимать PDF
\usepackage{pdflscape} % для возможности альбомного размещения некоторых страниц
\usepackage[pdftex]{hyperref}
% настройка ссылок в оглавлении для pdf формата
\hypersetup{unicode=true,
    pdftitle={Лабораторная робота №3},
    pdfauthor={Погода Михаил},
    pdfcreator={pdflatex},
    pdfsubject={},
    pdfborder    = {0 0 0},
    bookmarksopen,
    bookmarksnumbered,
    bookmarksopenlevel = 2,
    pdfkeywords={},
    colorlinks=true, % установка цвета ссылок в оглавлении
    citecolor=black,
    filecolor=black,
    linkcolor=black,
    urlcolor=blue
}

\usepackage{amsmath}
\usepackage{amssymb}

\usepackage{listings}
\lstloadlanguages{Python,C++}
\lstset{language=Python,frame=tb,basicstyle=\small,commentstyle=\itshape,extendedchars=false}

\usepackage{multicol}

\begin{document}
\begin{titlepage}
    \begin{center}
        \MakeUppercase{Министерство образования и науки Украины}\\
        \MakeUppercase{Национальный технический университет Украины}\\
        \MakeUppercase{<<Киевский политехнический институт>>}\\
        \vspace*{2em}

        Факультет прикладной математики\\
        Кафедра прикладной математики

        \vfill

        \MakeUppercase{Построение таблицы символов}\\
        \vspace*{2em}

        Отчёт по лабораторной работе \textnumero~3\\
        по дисциплине:\\
        <<Системы автоматизации программирования>>
    \end{center}

    \vfill
    \hfill\begin{minipage}{0.3\textwidth}
        Выполнил студент\\
        группы КМ-31м\\
        Погода~М.\,В.
    \end{minipage}

    \vfill
    \begin{center}
        Киев\\
        2014
    \end{center}
\end{titlepage}

\tableofcontents

\chapter{Постановка задачи}
\label{chap:first}
    В рамках этой лабораторной работы необходимо организовать построение таблицы символов к синтаксическому
    анализатору подмножества языка программирования \texttt{Python}, который был реализован в прошлой лабораторной
    работе.

    С использование таблицы символов необходимо провести семантический анализ кода.
\chapter{Организация таблицы символов и её компонентов}
\label{chap:second}
    Для организации работы используются следующая иерархия таблиц символов:
    \begin{enumerate}
        \item Глобальная таблица имён.
            Хранит имена объявленных классов и не даёт повторно объявить класс.
        \item Таблица имён класса.
            Хранит имена объявленных методов внутри класса.
            Не даёт объявить метод, имеющий имя, совпадающее с другим методом внутри этого класса, а также с классами,
            объявленными ранее.
        \item Локальная таблица имён.
            Хранит имена объявленных переменных.
            Позволяет выявить использование необъявленных переменных.
            Может хранить другие локальные таблицы имён, соответствующие блокам кода.
    \end{enumerate}
    \section{Выбранный метод поиска символов}
    \label{sec:formal}
        В данной работе используется поиска символов с помощью хеш"=таблиц.

        Поскольку язык является динамически"=типизированным, проверка наличия метода у объекта не производится во
        время семантического контроля.
        Эта проверка должна выполняться во время выполнения программы с помощью механизма поддержания работы.

        Для других символов производится контроль использования: переменная не может использована до тех пор, пока она
        не была объявлена.
        Объявленной считается переменная, которая либо является параметром метода, либо находилась с левой стороны от
        оператора присваивания.

\chapter{Метод построения таблицы символов}
    Синтаксический анализатор, который был разработан в рамках прошлой лабораторной работы, строит синтаксическое
    дерево, где каждой вершине, не являющейся листом, соответствует нетерминальный символ.

    Для построения таблицы символов достаточно осуществить обход дерева в глубину и выполнить семантическую процедуру
    в соответствии с типом вершины.
    Можно отметить, что модификация таблицы символов возможна только при обработке ограниченного числа вершин, а
    именно:
    \begin{itemize}
        \item \verb'{classdecl}'
        \item \verb'{methoddecl}'
        \item \verb'{if}'
        \item \verb'{while}'
        \item \verb'{new-identifier}'
    \end{itemize}

    Во время разбора синтаксического дерева, вызываются присоединённые семантические процедуры.
    Эти процедуры строят таблицы символов и производят проверки, описанные выше.
\chapter{Тестовые примеры}
\label{chap:fourth}
    \section{Пример без семантических ошибок}
    \label{sec:ex1}
        \lstinputlisting{../test1.py}
        Вывод программы:
        \begin{verbatim}
Class[[[ main
        Method[[[ run
                Parameters: self
                Local variables: m  a  c  d
        Method]]] run
Class]]] main
Class[[[ someclass
        Method[[[ bar
                Parameters: self  a  b
                Local variables: c
        Method]]] bar
        Method[[[ foo
                Parameters: self  x
                Local variables:
                Inner block[[[
                        Local variables:
                Inner block]]]
                Inner block[[[
                        Local variables:
                Inner block]]]
        Method]]] foo
Class]]] someclass
Class[[[ someotherclass
        Method[[[ foo
                Parameters: self  a  b
                Local variables: c  z
        Method]]] foo
Class]]] someotherclass

        \end{verbatim}

    \section{Пример с семантической ошибкой}
    \label{sec:ex2}
        \begin{lstlisting}
class main:
    def run( self ):
        a = b + c
        run = a

main().run()
        \end{lstlisting}
\chapter{Выводы}
\label{chap:concl}
    Для семантической корректности программы необходимо учесть для каждого синтаксического элемента, какие необходимо
    предусловия для семантической корректности выражения, а также какие изменения в таблицу символов может принести
    этот синтаксический элемент.

    Построение таблицы символов было встроено в синтаксический анализатор, т.к.\ все необходимые данные можно получить
    за один проход по исходному тексту.
\chapter*{Приложение. Исходный текст}
    \lstset{language=C++,basicstyle=\scriptsize}
    \lstinputlisting{../program_tree.cxx}
\end{document}
