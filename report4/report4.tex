% vim:spelllang=ru,en
\documentclass[a4paper,12pt,notitlepage,pdftex]{scrreprt}

\usepackage{cmap} % чтобы работал поиск по PDF
\usepackage[utf8]{inputenc}
\usepackage[english,russian]{babel}
\usepackage[T2A]{fontenc}
\usepackage{concrete}

\usepackage[pdftex]{graphicx}
\usepackage{subfig}

\pdfcompresslevel=9 % сжимать PDF
\usepackage{pdflscape} % для возможности альбомного размещения некоторых страниц
\usepackage[pdftex]{hyperref}
% настройка ссылок в оглавлении для pdf формата
\hypersetup{unicode=true,
    pdftitle={Лабораторная робота №4},
    pdfauthor={Погода Михаил},
    pdfcreator={pdflatex},
    pdfsubject={},
    pdfborder    = {0 0 0},
    bookmarksopen,
    bookmarksnumbered,
    bookmarksopenlevel = 2,
    pdfkeywords={},
    colorlinks=true, % установка цвета ссылок в оглавлении
    citecolor=black,
    filecolor=black,
    linkcolor=black,
    urlcolor=blue
}

\usepackage{amsmath}
\usepackage{amssymb}

\usepackage{listings}
\lstloadlanguages{Python,C++}
\lstset{language=Python,frame=tb,basicstyle=\small,commentstyle=\itshape,extendedchars=false}

\usepackage{multicol}

\begin{document}
\begin{titlepage}
    \begin{center}
        \MakeUppercase{Министерство образования и науки Украины}\\
        \MakeUppercase{Национальный технический университет Украины}\\
        \MakeUppercase{<<Киевский политехнический институт>>}\\
        \vspace*{2em}

        Факультет прикладной математики\\
        Кафедра прикладной математики

        \vfill

        \MakeUppercase{Генератор промежуточного кода}\\
        \vspace*{2em}

        Отчёт по лабораторной работе \textnumero~4\\
        по дисциплине:\\
        <<Системы автоматизации программирования>>
    \end{center}

    \vfill
    \hfill\begin{minipage}{0.3\textwidth}
        Выполнил студент\\
        группы КМ-31м\\
        Погода~М.\,В.
    \end{minipage}

    \vfill
    \begin{center}
        Киев\\
        2014
    \end{center}
\end{titlepage}

\tableofcontents

\chapter{Постановка задачи}\label{chap:first}
    В рамках этой лабораторной работы необходимо организовать схему перевода основных синтаксических конструкций
    подмножества языка программирования \texttt{Python}, определить таблицу базовых операций и реализовать
    соответствующий программный модуль.

    В качестве схемы перевода необходимо использовать триады.
\chapter{Базовые операции промежуточного кода}\label{chap:second}
    \paragraph{LOAD} Загружает данные из переменной, переданной как первый аргумент, в память.
    \paragraph{GET} Загружает данные из константы, переданной как первый аргумент, в память.
    \paragraph{Бинарные операции}
        Все бинарные операции представляются в виде \verb"{op} {addr-arg1} {addr-arg2}", где \verb"{op}" принимает
        одно из следующих значений:
    \begin{multicols}{4}
        \begin{itemize}
            \item \verb"+"
            \item \verb"-"
            \item \verb"*"
            \item \verb"/"
            \item \verb"=="
            \item \verb"!="
            \item \verb"<"
            \item \verb">"
        \end{itemize}
    \end{multicols}
    \paragraph{CREATE} Создаёт экземпляр класса, переданной как первый аргумент, в память.
    \paragraph{PUSH} Заносит данные, находящиеся по адресу, переданные как первый аргумент, в вершину стека.
    \paragraph{POP} Достаёт данные из вершины стека.
    \paragraph{STORE} Заносит данные, находящиеся по адресу, переданному как первый аргумент в переменную,
    переданную как второй аргумент.
    \paragraph{JMP} Безусловный переход.
    \paragraph{JNE} Условный переход если первый аргумент ложен.
    \paragraph{RET} Возврат из метода.
    \paragraph{CALL} Вызов метода у объекта.
\chapter{Организация генерации кода}
    Синтаксический анализатором строится синтаксическое дерево, каждый из узлов которого может иметь свою
    присоединённую процедуру в зависимости от типа вершины.

    По сути, генерация кода начинается с вызова присоединённой процедуры у вершины дерева, которая в свою очередь
    вызывает в необходимом порядке присоединённые процедуры у своих связей.
\chapter{Тестовые примеры}\label{chap:fourth}
        \lstinputlisting{../test1.py}
        \paragraph{Вывод программы}
        \begin{verbatim}
1       Class:  someclass       <<<>>>
2       Method:         foo     <<<>>>
3       POP
4       STORE   <3>     self
5       POP
6       STORE   <5>     x
7       LOAD    x
8       GET     5
9       ==      <7>     <8>
10      JNE     <9>     <15>
11      GET     «foo»
12      PUSH    <11>
13      CALL    SYSTEM  WRITE
14      JMP     <18>
15      GET     «bar»
16      PUSH    <15>
17      CALL    SYSTEM  WRITE
18      RET
19      Method:         bar     <<<>>>
20      POP
21      STORE   <20>    self
22      POP
23      STORE   <22>    a
24      POP
25      STORE   <24>    b
26      LOAD    b
27      LOAD    a
28      -       <26>    <27>
29      STORE   <28>    c
30      LOAD    c
31      PUSH    <30>
32      RET
33      RET
34      Class:  someotherclass  <<<>>>
35      Method:         foo     <<<>>>
36      POP
37      STORE   <36>    self
38      POP
39      STORE   <38>    a
40      POP
41      STORE   <40>    b
42      CREATE  someclass
43      STORE   <42>    c
44      LOAD    c
45      LOAD    a
46      PUSH    <45>
47      LOAD    b
48      PUSH    <47>
49      CALL    <44>    bar
50      POP
51      STORE   <50>    z
52      LOAD    c
53      LOAD    z
54      PUSH    <53>
55      CALL    <52>    foo
56      POP
57      RET
58      Class:  main    <<<>>>
59      Method:         run     <<<>>>
60      POP
61      STORE   <60>    self
62      CREATE  someotherclass
63      STORE   <62>    m
64      LOAD    m
65      GET     5
66      PUSH    <65>
67      GET     10
68      PUSH    <67>
69      CALL    <64>    foo
70      POP
71      LOAD    m
72      GET     3
73      PUSH    <72>
74      GET     6
75      PUSH    <74>
76      CALL    <71>    foo
77      POP
78      GET     1.53
79      STORE   <78>    a
80      GET     «asdd{}fgkjhns sdf asd »
81      STORE   <80>    c
82      GET     false
83      STORE   <82>    d
84      RET
85      CREATE  main
86      CALL    <85>    run
87      POP
        \end{verbatim}

\chapter{Выводы}\label{chap:concl}
    В данной лабораторной работе была рассмотрена генерация промежуточного кода в виде триад для синтаксического
    дерева, построенного синтаксический анализатором языка программирования Python в рамках второй лабораторной
    работы.

    Стоит отметить, что инструмент поддержания выполнения кода должен проверять наличие методов у объектов, поскольку
    язык Python является динамически-типизированным и нельзя определить, какие методы есть у объекта во время
    синтаксического разбора.
\chapter*{Приложение. Исходный текст}
    \lstset{language=C++,basicstyle=\scriptsize}
    \lstinputlisting{../main.cxx}
\end{document}
